\documentclass[pdftex,10pt,xcolor=svgnames]{beamer}

\mode<presentation>
{
  \usetheme{boxes}
  \usecolortheme[named=MidnightBlue]{structure}
  %\setbeamercolor{normal text}{bg=NavajoWhite!20}
  \usefonttheme{serif}
  \setbeamertemplate{navigation symbols}{}
  % Show frame number and author name in footline
  \setbeamertemplate{footline}[frame number]
  \addtobeamertemplate{footline}{\quad\textcolor{gray}{David Duvenaud and James Lloyd}}{}
  % Set frame titles in small capitals
  \setbeamerfont{frametitle}{shape=\scshape,family=\rmfamily}
  \setbeamercolor{frametitle}{bg=gray!60!white,fg=black}
  % Alerted text: blue (uncomment second line if theme sets alerted text to bold)
  \setbeamercolor{alerted text}{fg=blue}
  %\setbeamerfont*{alerted text}{}
  \setbeamertemplate{bibliography item}[text] %{\hbox{\donotcoloroutermaths$\blacktriangleright$}}
  \setbeamertemplate{bibliography entry title}{}
  \setbeamertemplate{bibliography entry author}{}
  \setbeamertemplate{bibliography entry note}{}
  \setbeamertemplate{bibliography entry location}{}

}
\usepackage[english]{babel}
\usepackage[latin1]{inputenc}
\usepackage{times}
\usepackage[T1]{fontenc}
\usepackage{hyperref}
\usepackage{multimedia}
\usepackage{eepic}
\usepackage{graphicx}
%\usepackage[nohug]{latexinclude/diagrams}
\usepackage{tikz}
\usetikzlibrary{calc}

%% \newcommand{\footlineextra}[1]{
%%     \begin{tikzpicture}[remember picture,overlay]
%%         \node[yshift=1.5ex,anchor=south east] at (current page.south east)
%% {#1};
%%     \end{tikzpicture}
%% }

\newcommand{\footlineextra}[1]{
    \begin{tikzpicture}[remember picture,overlay]
        \node[xshift=-5ex,yshift=-0.5ex,anchor=south east] at (current page.south east)
             {\mbox{\tiny \textcolor{MidnightBlue}{#1}}};
    \end{tikzpicture}
}

\def\sectionframe#1{
  {
    \setbeamertemplate{footline}{\empty}
    \begin{frame}{}
      \begin{center}
        \huge\sc #1
      \end{center}
    \end{frame}
  }
}



\usecolortheme{default}
% colours
\xdefinecolor{Black}{rgb}{0,0,0}
\xdefinecolor{White}{rgb}{1,1,1}
\xdefinecolor{DarkBlue}{rgb}{0,0,.7}
\xdefinecolor{DarkRed}{rgb}{.7,0,0}
\xdefinecolor{Red}{rgb}{.85,0,0}
\xdefinecolor{DarkGreen}{rgb}{0,.7,0}
\xdefinecolor{DarkMagenta}{rgb}{.6,0,.6}
\def\Black{\textcolor{Black}}
\def\White{\textcolor{White}}
\def\Blue{\textcolor{DarkBlue}}
\def\Magenta{\textcolor{DarkMagenta}}
\def\Red{\textcolor{Red}}
\def\Green{\textcolor{DarkGreen}}

\usepackage{alltt}

\title[] % (optional, use only with long paper titles)
{Introduction to probabilistic programming}

\author % (optional, use only with lots of authors)
{David Duvenaud and James Lloyd}
% - Use the \inst{?} command only if the authors have different
%   affiliation.

\institute[] % (optional, but mostly needed)
{University of Cambridge}
% - Use the \inst command only if there are several affiliations.
% - Keep it simple, no one is interested in your street address.

\date % (optional)
{\empty}

\subject{Talks}

\usetikzlibrary{shapes.geometric,arrows,chains,matrix,positioning,scopes}
 \makeatletter
 \tikzset{join/.code=\tikzset{after node path={%
       \ifx\tikzchainprevious\pgfutil@empty\else(\tikzchainprevious)%
       edge[every join]#1(\tikzchaincurrent)\fi}}
 }
 \tikzset{>=stealth',every on chain/.append style={join},
   every join/.style={->}
 }

\tikzstyle{mybox} = [draw=white, rectangle]
\usepackage{ifthen}
\usepackage{booktabs}

% Custom definitions
\def\simiid{\sim_{\mbox{\tiny iid}}}

\begin{document}

\small

%% { 
%%   \setbeamertemplate{footline}{\empty}
%%   \begin{frame}
%%     \titlepage
%%   \end{frame}
%% }
\renewcommand{\inserttotalframenumber}{11}

\theoremstyle{plain}

\def\ie{i.e.\ }
\def\eg{e.g.\ }
\def\indicator{\mathbb{I}}
\def\mean#1{\mathbb{E}[#1]}
\def\bigmean#1{\mathbb{E}\bigl[#1\bigr]}
\def\Bigmean#1{\mathbb{E}\Bigl[#1\Bigr]}
\def\cyl{\mathcal{Z}}
\def\eqae{=_{\mbox{\tiny a.e.}}}
\def\wrt{w.r.t.\ }
\def\ae{a.e.\ }
\def\equas{=_{\mbox{\tiny a.s.}}}
\def\equae{=_{\mbox{\tiny a.e.}}}
\def\iid{i.i.d.\ }
\def\Iid{I.i.d.\ }
%\def\inclusion{\jmath}
\def\inclusion{\mathcal{J}}
\def\inclusionX{\inclusion_{\xspace}}
\def\wstar{weak$^{\ast}$ }
% Symmetric difference
\def\symmdiff{\!\vartriangle\!}


% Indices

\def\indI{\mbox{\tiny I}}
\def\indJ{\mbox{\tiny J}}
\def\indK{\mbox{\tiny K}}
\def\indJI{\mbox{\tiny J$\setminus$I}}
\def\indE{\mbox{\tiny E}}
\def\indF{\mbox{\tiny F}}
\def\indD{\mbox{\tiny D}}
\def\indi{\mbox{\tiny{\{i\}}}}
\def\ind#1{\mbox{\tiny #1}}
\def\power{\mathcal{F}}
\def\powerD{\power(D)}
\def\powerE{\power(E)}
\def\powerL{\power(L)}
\def\parts{\mathcal{H}}
\def\partsQ{\parts(\mathcal{Q})}
\def\partsn{\parts[n]}
\def\partsN{\parts_{\infty}(\mathbb{N})}

% Spaces

\def\abstspace{\Omega}
\def\xspace{\mathcal{X}}
\def\yspace{\mathcal{Y}}
\def\tspace{\mathcal{T}}
\def\xspaceI{\xspace_{\indI}}
\def\xspaceJ{\xspace_{\indJ}}
\def\xspaceD{\xspace_{\indD}}
\def\xspaceE{\xspace_{\indE}}
\def\tspaceI{\tspace_{\indI}}
\def\tspaceJ{\tspace_{\indJ}}
\def\tspaceD{\tspace_{\indD}}
\def\tspaceE{\tspace_{\indE}}
\def\txspace{\tilde{\xspace}}
\def\yspaceI{\yspace_{\indI}}
\def\yspaceJ{\yspace_{\indJ}}
\def\yspaceD{\yspace_{\indD}}
\def\yspaceE{\yspace_{\indE}}
\def\txspace{\tilde{\xspace}}
\def\ttspace{\tilde{\tspace}}
\def\xI{x_{\indI}}
\def\xJ{x_{\indJ}}
\def\xD{x_{\indD}}
\def\xE{x_{\indE}}
\def\tImage{\Gamma}
\def\simp{\triangle}
\def\simpI{\simp_{\indI}}
\def\simpJ{\simp_{\indJ}}

\def\AI{A_{\indI}}
\def\AJ{A_{\indJ}}
\def\AD{A_{\indD}}
\def\AE{A_{\indE}}


%Space of Prob Measures
\def\pMeas{M}
%Space of Contents
\def\fMeas{N}
%Space of cont fcts
\def\cfspace{C}
%Hilbert space
\def\hilbert{\mathcal{L}^2}


\def\borelV{\borel_{V}}

% Set systems

\def\borel{\mathcal{B}}
\def\top{\mbox{Top}}

\def\borelI{\borel_{\indI}}
\def\borelJ{\borel_{\indJ}}
\def\borelD{\borel_{\indD}}
\def\borelE{\borel_{\indE}}
\def\tborel{\tilde{\borel}}
\def\abstfield{\mathcal{A}}
\def\field{\mathcal{C}}
\def\fieldI{\field_{\indI}}
\def\fieldJ{\field_{\indJ}}
\def\fieldK{\field_{\indK}}
\def\fieldD{\field_{\indD}}
\def\fieldE{\field_{\indE}}
\def\tfield{\tilde{\mathcal{C}}}
\def\Sfield{\mathcal{S}}
\def\SfieldI{\mathcal{S}_{\indI}}
\def\SfieldJ{\mathcal{S}_{\indJ}}
\def\SfieldD{\mathcal{S}_{\indD}}
\def\tSfield{\tilde{\mathcal{S}}}
\def\borelx{\borel_x}
\def\tborelx{\tborel_x}
\def\borelgamma{\tborel_{\tImage}}
%\def\borelth{\borel_{\theta}}
\def\borely{\borel_{y}}
%\def\borelT{\borel_t}
\def\borelT{\borel_{\tspace}}
\def\borelS{\borel_s}
\def\topI{\top_{\indI}}
\def\topJ{\top_{\indJ}}
\def\topD{\top_{\indD}}
\def\topE{\top_{\indE}}
\def\topV{\top_V}
\def\topws{\top_{\text{ws}}}
\def\topcc{\top_{\text{c}}}
\def\borelXI{\borel(\xspaceI)}
\def\borelXD{\borel(\xspaceD)}
\def\tborelX{\borel(\txspace)}
\def\borelTI{\borel(\tspaceI)}
\def\borelTD{\borel(\tspaceD)}
\def\tborelT{\borel(\ttspace)}


% Maps

\def\XI{X_{\indI}}
\def\Xi{X_{\ind{i}}}
\def\Xj{X_{\ind{j}}}
\def\ThetaI{\Theta_{\indI}}
\def\XJ{X_{\indJ}}
\def\ThetaJ{\Theta_{\indJ}}
\def\XD{X_{\indD}}
\def\ThetaD{\Theta_{\indD}}
\def\XE{X_{\indE}}
\def\ThetaE{\Theta_{\indE}}
\def\tX{\tilde{X}}
\def\tTheta{\tilde{\Theta}}

\def\SI{S_{\indI}}
\def\TI{T_{\indI}}

\def\rest{\phi}
\def\restD{\rest_{\indD}}
\def\restI{\rest_{\indI}}
\def\restJ{\rest_{\indJ}}
\def\restDI{\rest^{\indD}_{\indI}}
\def\inclusionD{\inclusion_{\indD}}
\def\inclusionE{\inclusion_{\indE}}
\def\projector{\mbox{pr}}
\def\projectorD{\projector_{\indD}}
\def\projectorI{\projector_{\indI}}
\def\projectorJI{\pi_{\indJ\indI}}
\def\indicator{\mathbb{I}}

% Projective systems

\def\po{\preceq}
\def\famD#1{{\lbrace #1 \rbrace}_{\indD}}
\def\famE#1{{\lbrace #1 \rbrace}_{\ind{I$\in$}\indE}}
\def\fJI{f_{\indJ\indI}}
\def\fKI{f_{\indK\indI}}
\def\fKJ{f_{\indK\indJ}}
\def\fII{f_{\indI\indI}}
\def\fI{f_{\indI}}
\def\fJ{f_{\indJ}}
\def\fK{f_{\indK}}
\def\fD{f_{\indD}}
\def\fDI{f^{\indD}_{\indI}}
\def\fDK{f^{\indD}_{\indK}}
\def\gJI{g_{\indJ\indI}}
\def\gI{g_{\indI}}
\def\gJ{g_{\indJ}}
\def\gD{g_{\indD}}
\def\hJI{h_{\indJ\indI}}
\def\hI{h_{\indI}}
\def\hJ{h_{\indJ}}
\def\hE{h_{\indE}}
\def\plim{\varprojlim}

% Measure and Conditionals

\def\abstmeasure{\mathbb{P}}
\def\P{P}
\def\PI{P_{\indI}}
\def\PJ{P_{\indJ}}
\def\PD{P_{\indD}}
\def\PE{P_{\indE}}
\def\PX{P_{\mbox{X}}}
\def\PTh{P_{\mbox{\Theta}}}
\def\PXI{P_{\XI}}
\def\PThI{P_{\mbox{\Theta}}}
\def\PXJ{P_{\mbox{X}}}
\def\PThJ{P_{\mbox{\Theta}}}
\def\PXD{P_{\mbox{X}}}
\def\PThD{P_{\mbox{\Theta}}}
\def\PXE{P_{\mbox{X}}}
\def\PThE{P_{\mbox{\Theta}}}
\def\tP{\tilde{P}}
\def\tPX{\tilde{P}_X}
\def\tPTh{\tilde{P}_{\Theta}}







\def\SI{S_{\indI}}
\def\SJ{S_{\indJ}}

\def\tk{\tilde{k}}
\def\kI{k_{\indI}}

\def\postkernel{k}
\def\indctr{\mathbbm{1}}
\def\sp#1{\left<#1\right>}


%Mallows
\def\Sr{\mathbb{S}_r}
\def\Sinf{\mathbb{S}_{\infty}}
\def\Sbar{\bar{\mathbb{S}}}
\def\DP#1{\mbox{DP}\left( #1 \right)}
\def\GP#1{\mbox{GP}\left( #1 \right)}
\def\x{\mathbf{x}}
\def\y{\mathbf{y}}



\def\tyspace{\tilde{\yspace}}
\def\tF{\tilde{F}}
\def\tT{\tilde{T}}
\def\tmodel{\tilde{\model}}
\def\tnu{\tilde{\nu}}


\def\PTheta{P^{\theta}}
\def\FTheta{F^{\theta}}
\def\TTheta{T^{\theta}}
\def\borelY{\borel_{\yspace}}

\def\PX{P^{x}}
\def\PXI{\PX_{\indI}}
\def\PXJ{\PX_{\indJ}}
\def\PXD{\PX_{\indD}}
\def\PThetaI{\PTheta_{\indI}}
\def\PThetaD{\PTheta_{\indD}}
\def\YI{Y_{\indI}}
\def\YJ{Y_{\indJ}}
\def\YD{Y_{\indD}}
\def\Tn{T^{(n)}}
\def\indexspace{\mathcal{W}}
\def\tyspace{\tilde{\yspace}}
\def\tY{\tilde{Y}}
\def\inclusionT{\inclusion_{\tspace}}
\def\tPTheta{\tilde{P}^{\theta}}
\def\tTn{\tilde{T}^{(n)}}
\def\inclusionY{\inclusion_{\yspace}}

\def\tyspace{\tilde{\yspace}}
\def\tF{\tilde{F}}
\def\tT{\tilde{T}}
\def\tmodel{\tilde{\model}}
\def\tnu{\tilde{\nu}}
\def\tOmega{\tilde{\abstspace}}
\def\tabstmeasure{\tilde{\abstmeasure}}
\def\model{\mathcal{P}}

\def\tf{\tilde{f}}
\def\tx{\tilde{x}}
\def\Dom{\mbox{Dom}}
\def\ty{\tilde{y}}


\begin{frame}
  \begin{block}{}
    \titlepage
  \end{block}
  \begin{center}
    {\bf Thanks to}\\
    Daniel M Roy (Cambridge)\\
    Roger Grosse (MIT)
  \end{center}
\end{frame}

\begin{frame}{How to write a Bayesian modeling paper}
  \begin{block}{}
    \begin{enumerate}
      \item Write down a generative model in an afternoon
      \vspace{\baselineskip}
      \vspace{\baselineskip}
      \item Get 2 grad students to implement inference for a month
      \vspace{\baselineskip}
      \vspace{\baselineskip}
      \item Use technical details of inference to pad half of the paper
    \end{enumerate}
  \end{block}
\end{frame}

\begin{frame}{Can we do better?}
  \begin{block}{Example: Graphical Models}  
  \end{block}
      \begin{block}{Application Papers}
      \begin{enumerate}
        \item Write down a graphical model
        \item Perform inference using general-purpose software
        \item Apply to some new problem
      \end{enumerate}
    \end{block}
      \begin{block}{Inference papers}
      \begin{enumerate}
        \item Identify common structures in graphical models (e.g. chains)
        \item Develop efficient inference method
        \item Implement in a general-purpose software package
      \end{enumerate}
    \end{block}  
  \begin{block}{}
      \vspace{-2\baselineskip}
  \large
    \begin{center}
    {
      {Modeling and inference have been disentangled}
    }	
    \end{center}
  \end{block}
\end{frame}



\begin{frame}{Expressivity}
  \begin{block}{Not all models are graphical models}  
  What is the largest class of models available?
  \end{block}
      \begin{block}{Probabilistic Programs}
      \begin{enumerate}
        \item A probabilistic program (PP) is any program that can depend on random choices.  Can be written in any language that has a (P)RNG.
        \item You can specify any (computable) prior by simply writing down a PP that generates samples
        \item Any PP implicitly defines a distribution over execution traces
      \end{enumerate}
    \end{block}
    TODO: show a program and a histogram of its output
\end{frame}

\begin{frame}{Probabilistic Programs vs Probabilistic Programming}
  \begin{block}{Once we've defined a prior, what do we want to do with it?}  
  The PP defines $P(D,N,H)$, we choose D to be the subset of variables we observe, H the set of variables we're interested in, and N the set of variables that we're not interested in, so we'll integrate them out.  We want to get to $P(H|D)$
  \end{block}
      \begin{block}{Probabilistic Programming}
      \begin{enumerate}
        \item Usually refers to doing inference when a PP specifies your prior.
        \item 
      \end{enumerate}
    \end{block}
    TODO: Show the two possibilities of conditioning in th eprevious program
\end{frame}

\begin{frame}{Can we develop generic inference for all PPs?}
Yes - rejection sampling.
But can we be more efficient whilst being generic?

Yes. MCMC over execution traces.
\end{frame}

\begin{frame}{PP via MCMC}
  \begin{block}{}
    Following Wingate et alia we represent the unconditioned PP as a parameterless function $f$
    \newline
    
    Evaluating $f$ results in random choices which are denoted as
    \begin{equation*}
      x_k = f_{k|x_1,\ldots,x_{k-1}} \sim p_{t_k}(.|\theta_{k},x_1,\ldots,x_{k-1}).
    \end{equation*}
    
    The density / probability of a particular evaluation is then
    \begin{equation*}
      p(x) = \prod_{k=1}^K p_{t_k}(x_k|\theta_{k},x_1,\ldots,x_{k-1}).
    \end{equation*}
    
    We then perform MCMC over the $x_k$ \ie the execution trace.
    
  \end{block}
\end{frame}

\begin{frame}{MCMC over execution traces}
  \begin{enumerate}
    \item Select a random $x_k = f_k$ in the execution trace
    \item Propose a new value $x_k' \sim K_{t_k}(.|x_k,\theta_k)$
    \item Run the program to determine all subsequent choices $(x_l' : l > k)$, reusing current choices where possible
    \item Propose moving from the state $(x_1,\ldots,x_K)$ to $(x_1,\ldots,x_{k-1},x_k',\ldots,x_{K'}')$
    \item Accept the change with the appropriate reversible jump MCMC acceptance probability, this includes terms like
    \begin{enumerate}
      \item $K_{t_k}(x_k'|x_k,\theta_k),\,K_{t_k}(x_k|x_k',\theta_k),\,p_{t_k}(x_k|\theta_{k},x_1,\ldots,x_{k-1})$
      \item $\prod_{i=k}^K p_{t_i}(x_i|\theta_{i},x_1,\ldots,x_{i-1}),\,\prod_{i=k}^{K'} p_{t_i'}(x_i'|\theta_{i}',x_1,\ldots,x_{k-1},x_k',\ldots,x_{i-1}')$
      \item \ie $\frac{K_{t_k}(x_k|x_k',\theta_k)\prod_{i=k}^{K'} p_{t_i'}(x_i'|\theta_{i}',x_1,\ldots,x_{k-1},x_k',\ldots,x_{i-1}')}{K_{t_k}(x_k'|x_k,\theta_k)\prod_{i=k}^K p_{t_i}(x_i|\theta_{i},x_1,\ldots,x_{i-1})}$
    \end{enumerate}
  \end{enumerate}
\end{frame}

\begin{frame}{Worked example}
\end{frame}

\begin{frame}{Further Generi inference methods}
\eg HMC, parallel tempering, etc.
Remember graphical models (fancy algorithms that work in certain model classes)
\end{frame}

\begin{frame}{PP timeline}

Infer.net?
\end{frame}

\begin{frame}{Example: Mixture of Gaussians}-
  \begin{columns}
    \begin{column}{.5\textwidth}
      \begin{block}{Generative model}
        \begin{eqnarray*}
          (\mu_i)_{i=1\ldots k} & \simiid & \mathcal{N}(0, 1) \\
          (\pi_i)_{i=1\ldots k} & \sim & \textrm{Dir}(\alpha) \\
          \Theta & := & \sum_{i=1}^k \pi_i \delta_{\mu_i} \\
          (\theta_i)_{i=1\ldots n} & \simiid & \Theta \\
          (x_i)_{i=1\ldots n} & \simiid & \mathcal{N}(\theta_i, 1)
        \end{eqnarray*}
      \end{block}
    \end{column}
    \begin{column}{.5\textwidth}
      \begin{block}{(Pseudo) MATLAB code}
        \vspace{0.75\baselineskip}
        \begin{alltt}
          mu = randn(k,1);

          pi = dirichlet(k, alpha);


          for i = 1:n
            
          \ \ theta = mu(mnrnd(1,pi));
          
          \ \ x(i) \ = theta + randn;
            
          end
        \end{alltt}
        \vspace{0.75\baselineskip}
      \end{block}
    \end{column}
  \end{columns}
\end{frame}

\begin{frame}{Example: Infinite mixture of Gaussians}
  \begin{block}{Change to generative model}
    \begin{equation*}
      \Theta := \sum_{i=1}^k \pi_i \delta_{\mu_i} \to \Theta \sim \textrm{DP}(\alpha, \mathcal{N}(0,1))
    \end{equation*}
  \end{block}
  \begin{block}{(Pseudo) MATLAB code - stick breaking construction}
    \begin{alltt}
      sticks = []; atoms = [];
      
      for i = 1:n
      
      \ \ p = rand;
      
      \ \ while p > sum(sticks)
      
      \ \ \ \ sticks(end+1) = (1-sum(sticks)) * betarnd(1, alpha);
      
      \ \ \ \ atoms(end+1) \ = randn;
      
      \ \ end
      
      \ \ theta(i) = atoms(find(cumsum(sticks)>=p, 1, 'first'));
      
      end

      x = theta' + randn(n, 1);
    \end{alltt}
  \end{block}
\end{frame}

\begin{frame}{Stochastic memoisation}
  \begin{block}{}
    \begin{enumerate}
      \item The stick breaking construction can be applied to any base measure
      \vspace{\baselineskip}
      \vspace{\baselineskip}
      \item Church provides the function \texttt{DPmem} that takes any base measure sampling function and returns a function that samples from a sample from the corresponding Dirichlet process
      \vspace{\baselineskip}
      \vspace{\baselineskip}
      \item This allows easy specification of many nonparametric models \eg HDP based models
    \end{enumerate}
  \end{block}
\end{frame}

\end{document}


